\documentclass{beamer}
\mode<presentation>
\usetheme{Frankfurt}
\usecolortheme{seahorse}
\usecolortheme{rose}
\setbeamercovered{transparent}

\usepackage[english]{babel}
\usepackage{tikz}
\usepackage{tikz-3dplot}
\usetikzlibrary{positioning,3d,calc}

\makeatletter
\newenvironment<>{proofs}[1][\proofname]{
	\par
	\def\insertproofname{#1\@addpunct{.}}
	\usebeamertemplate{proof begin}#2
}{
	\usebeamertemplate{proof end}
}
\newenvironment<>{proofc}{
	\setbeamertemplate{proof begin}{\begin{block}{\proofname\ (cont.)}}
	\par
	\usebeamertemplate{proof begin}
}{
	\usebeamertemplate{proof end}
}
\newenvironment<>{proofe}{
	\par
	\pushQED{\qed}
	\setbeamertemplate{proof begin}{\begin{block}{\proofname\ (cont.)}}
	\usebeamertemplate{proof begin}
}{
	\popQED\usebeamertemplate{proof end}
}
\makeatother

\title[What is... the mixed volume?]{What is... it like to lower the mixed volume?}
\author{Dimitris Bogiokas}
\institute{Freie Universit\"at Berlin}
\date{BMS Friday - What is...? Seminar\\08. Feb. 2019}
\subject{Mathematics}

% \pgfdeclareimage[height=0.5cm]{university-logo}{university-logo-filename}
% \logo{\pgfuseimage{university-logo}}

\AtBeginSubsection[]{
  \begin{frame}<beamer>{Outline}
    \tableofcontents[currentsection,currentsubsection]
  \end{frame}
}
%\beamerdefaultoverlayspecification{<+->}

\begin{document}

\begin{frame}
  \titlepage
\end{frame}

\begin{frame}{Outline}
  \tableofcontents[pausesections]
\end{frame}

\section{Setup}
\begin{frame}{Ingredients}
\begin{itemize}
\item<2-> The set $\mathcal{K}_n$ of all \alert<2>{convex}, \alert<2>{compact} sets in $\mathbb{R}^n$.
\item<3-> The set $\mathcal{P}_n\subseteq\mathcal{K}_n$ of all convex \alert<3>{polytopes} in $\mathbb{R}^n$.
\item<4-> The \alert<4>{volume} $V_n(K)$ of some $K\in\mathcal{K}_n$.
\item<5-> The \alert<5>{stretching} $\lambda K$ of some $K\in\mathcal{K}_n$, by any $\lambda\geq0$.
\item<6-> The \alert<6>{Minkowski sum} $K+L$ of some $K,L\in\mathcal{K}_n$.
\pause
\end{itemize}
\end{frame}

\begin{frame}{The Minkowski Sum of Two Convex Bodies}
\begin{definition} For any two $K,L\in\mathcal{K}_n$, the \alert{Minkowski sum} $K+L\in\mathcal{K}_n$ is the set of every vector sum:
$$K+L:=\{u+v\in\mathbb{R}^n:u\in K,v\in L\}$$
\end{definition}
\vspace*{1.5em}
\begin{center}
\begin{tikzpicture}
\coordinate (A) at (0,0);
\coordinate (B) at (-3.5,1.3);
\filldraw[draw=red!50,fill=red!10] (A) arc (-90:30:1) arc (120:180:1.732) -- cycle;
\node at (-.4,.8) {$+$};
\filldraw[draw=blue!50,fill=blue!10] (B) --++(2,0) arc (90:-90:.5) -- cycle;
\node at (2,.8) {$=$};
\coordinate (AB) at (3,.3);
\filldraw[draw=blue!50!red!50,fill=blue!50!red!10] (AB) arc (180:120:1.732) -- ++(2,0) arc (90:30:.5) arc (30:-90:1.5) -- cycle;
\filldraw<2-14>[draw=red!50,fill=red!10,opacity=.9] (AB) arc (-90:30:1) arc (120:180:1.732) -- cycle;
\foreach\n in {2,3,...,8}{
	\def\a{-130 + 20*\n};
	\filldraw<\n>[blue!50,fill=blue!10,opacity=.9] (AB)++(90:1)++(\a:1) --++(2,0) arc (90:-90:.5) -- cycle;
}
\foreach\n in {9,10,...,14}{
	\def\a{40 + 10*\n};
	\filldraw<\n>[blue!50,fill=blue!10,opacity=.9] (AB)++(90:1)++(30:1)++(-60:1.732)++(\a:1.732) --++(2,0) arc (90:-90:.5) -- cycle;
}
\node<15> at (0,0) {};
\end{tikzpicture}
\end{center}
\end{frame}

\section{Talking About the Volumes}

\begin{frame}{The Brunn-Minkowski Inequality}
\begin{theorem}[Brunn '87, Minkowski '96]
Let $K,L\in\mathcal{K}_n$ be two compact, convex sets, then:
$$V_n(K+L)^{\frac{1}{n}}\geq V_n(K)^{\frac{1}{n}}+V_n(L)^{\frac{1}{n}}$$
\end{theorem}
\end{frame}
\begin{frame}{Minkowski Sum Increases the Volume}
\begin{corollary}
Let $K,L\in\mathcal{K}_n$ be two compact, convex sets, then:
$$V_n(K+L)\geq V_n(K)+V_n(L)$$
\end{corollary}
\begin{proof}<2->[Proof (by picture)]
\begin{center}
\begin{tikzpicture}
\coordinate (A) at (0,0);
\coordinate (B) at (-1.3,0);
\filldraw[fill=red!10,draw=red!50] (A) -- ++(1,1) arc (45:-135:.7071) -- cycle;
\node at (-.5,.5) {$+$};
\filldraw[fill=blue!10,draw=blue!50] (B) -- ++(0,1) -- ++(-.5,0) -- cycle;
\fill<3->[fill=red!60] (A)++(1,1) circle (2pt);
\fill<3->[fill=blue!60] (B) circle (2pt);
\node at (2,.5) {$=$};
\coordinate (AB) at (3,-.5);
\filldraw[fill=blue!50!red!10,draw=blue!50!red!50] (AB) -- ++(-.5,1) -- ++(1,1) -- ++(.5,0) arc (45:0:.7071) -- ++(0,-1) arc (0:-135:.7071) -- cycle;
\filldraw<4->[fill=red!10,draw=red!50,opacity=.9] (AB) -- ++(1,1) arc (45:-135:.7071) -- cycle;
\filldraw<4->[fill=blue!10,draw=blue!50,opacity=.9] (AB)++(1,1) -- ++(0,1) -- ++(-.5,0) -- cycle;
\fill<3->[fill=blue!50!red!60] (AB)++(1,1) circle (2pt);
\end{tikzpicture}
\end{center}
\end{proof}
\end{frame}

\begin{frame}{More Handwaving}
\begin{proofs}[Proof of Brunn-Minkowski Inequality]
For $K=[0,x_1]\times\cdots\times[0,x_n]$ and $L=[0,y_1]\times\cdots\times[0,y_n]$:
\begin{align*}
\frac{V_n(K)^{\frac{1}{n}}+V_n(L)^{\frac{1}{n}}}{V_n(K+L)^{\frac{1}{n}}}&=\left(\prod_{i=1}^n\frac{x_i}{x_i+y_i}\right)^{\frac{1}{n}}+\left(\prod_{i=1}^n\frac{y_i}{x_i+y_i}\right)^{\frac{1}{n}}\\
&\leq\frac{1}{n}\sum_{i=1}^n\frac{x_i}{x_i+y_i}+\frac{1}{n}\sum_{i=1}^n\frac{y_i}{x_i+y_i}\\
&=1
\end{align*}
\end{proofs}
\end{frame}

\begin{frame}{More Handwaving}
\begin{proofc}
For $K,L$ disjoint unions of open boxes:
\begin{center}
\begin{tikzpicture}
\coordinate (A) at (3,.5);
\coordinate (B) at (0,0);
\filldraw[fill=blue!10,draw=blue!50] (B) -- ++(1,0) -- ++(0,2) -- ++(-1,0) -- cycle;
\filldraw[fill=blue!10,draw=blue!50] (B)++(1,-.5) -- ++(1,0) -- ++(0,1) -- ++(-1,0) -- cycle;
\filldraw[fill=blue!10,draw=blue!50] (B)++(1,.5) -- ++(.5,0) -- ++(0,1) -- ++(-.5,0) -- cycle;
\node at (2.2,1.2) {$+$};
\filldraw<1-3>[fill=red!10,draw=red!50] (A) -- ++(3,0) -- ++(0,1) -- ++(-3,0) -- cycle;
\filldraw<1-3>[fill=red!10,draw=red!50] (A)++(3,.75) -- ++(0,1) -- ++(1.333,0) -- ++(0,-1) -- cycle;
\draw<2-3>[dashed] (-1,.5) -- (8.333,.5);
\node at (0,2.25) {};
\node<3-> at (-.5,1) {$\frac{4}{7}$};
\node<3-> at (-.5,0) {$\frac{3}{7}$};
\filldraw<4->[fill=red!10,draw=red!50] (A)++(0,-.333) -- ++(3,0) -- ++(0,1) -- ++(-3,0) -- cycle;
\filldraw<4->[fill=red!10,draw=red!50] (A)++(0,-.333)++(3,.75) -- ++(0,1) -- ++(1.333,0) -- ++(0,-1) -- cycle;
\draw<4->[dashed] (-1,.5) -- (8.333,.5);
\node<4-> at (.7,-.3) {$K_-$};
\node<4-> at (1.5,1.8) {$K_+$};
\node<4-> at (6.5,-.3) {$L_-$};
\node<4-> at (5.5,1.8) {$L_+$};
\end{tikzpicture}
\end{center}
\end{proofc}
\end{frame}

\begin{frame}{More Handwaving}
\begin{proofe}
\begin{align*}
V_n(K+L)&\geq V_n(K_-+L_-)+V_n(K_++L_+)\\
&\geq \left(V_n(K_-)^{\frac{1}{n}}+V_n(L_-)^{\frac{1}{n}}\right)^n+\left(V_n(K_+)^{\frac{1}{n}}+V_n(L_+)^{\frac{1}{n}}\right)^n\\
&=V_n(K_-)\left(1+\frac{V_n(L)^{\frac{1}{n}}}{V_n(K)^{\frac{1}{n}}}\right)^n+V_n(K_+)\left(1+\frac{V_n(L)^{\frac{1}{n}}}{V_n(K)^{\frac{1}{n}}}\right)^n\\
&=V_n(K)\left(1+\frac{V_n(L)^{\frac{1}{n}}}{V_n(K)^{\frac{1}{n}}}\right)^n=\left(V_n(K)^{\frac{1}{n}}+V_n(L)^{\frac{1}{n}}\right)^n
\end{align*}
For the general case, approximate with boxes.
\end{proofe}
\end{frame}

\begin{frame}{Find Highest Volume Under Constant Surface Area}
\begin{theorem}[Isoperimetric Inequality]
Let $K\in\mathcal{K}_n$ be a compact, convex set, then:
$$V_n(K)\leq V_n(rB^n)$$
where $B_n$ is the unit ball of dimension $n$ and $r$ is chosen so that:
$$V_{n-1}(K)=V_{n-1}(rB^n)$$
\end{theorem}
\pause
\begin{corollary}
An equivalent statement of the theorem is that for every $K\in\mathcal{K}_n$:
$$\frac{V_n(K)^{\frac{1}{n}}}{V_{n-1}(K)^{\frac{1}{n-1}}}\leq\frac{V_n(B^n)^{\frac{1}{n}}}{V_{n-1}(B^n)^{\frac{1}{n-1}}}$$
\end{corollary}
\end{frame}

\begin{frame}{But What is Surface Area?}
\begin{definition}
Let $K\in\mathcal{K}_n$. Then define the \alert{surface area} of $K$ to be the limit:
$$V_{n-1}(K):=\lim_{\varepsilon\to0^+}\frac{V_n(K_{\varepsilon})-V_n(K)}{\varepsilon}$$
where $K_{\varepsilon}:=K+\varepsilon B^n$.
\end{definition}
\pause
\begin{example}
\begin{align*}
V_{n-1}(B^n)&=\lim_{\varepsilon\to0^+}\frac{V_n\big((1+\varepsilon)B^n\big)-V_n(B^n)}{\varepsilon}\\
&=V_n(B^n)\lim_{\varepsilon\to0^+}\frac{(1+\varepsilon)^n-1}{\varepsilon}\\
&=nV_n(B^n)
\end{align*}
\end{example}
\end{frame}

\begin{frame}{Proof of Isoperimetric using B-M}
\begin{proof}[Proof of Isoperimetric Inequality]
Let $\varepsilon>0$. Then:
\begin{align*}
\frac{V_n(K+\varepsilon B^n)-V_n(K)}{\varepsilon}&\geq\frac{\left(V_n(K)^{\frac{1}{n}}+\varepsilon V_n(B^n)^{\frac{1}{n}}\right)^n-V_n(K)}{\varepsilon}\\
&=n\left(V_n(K)^{\frac{1}{n}}\right)^{n-1}V_n(B^n)^{\frac{1}{n}}+O(\varepsilon)
\end{align*}
Taking the limit on both sides, for $\varepsilon\to0^+$:
\begin{align*}
V_{n-1}(K)&\geq nV_n(K)^{\frac{n-1}{n}}V_n(B^n)^{\frac{1}{n}}\\
\Rightarrow \frac{V_n(K)^{\frac{1}{n}}}{V_{n-1}(K)^{\frac{1}{n-1}}}&\leq\frac{V_n(B^n)^{\frac{1}{n}}}{V_{n-1}(B^n)^{\frac{1}{n-1}}}
\qedhere
\end{align*}
\end{proof}
\end{frame}

\section{Mixed Volume}

\begin{frame}{Can We Speak In General About Linear Combinations Of Convex Bodies?}
\begin{example}
Let $K=[0,1]^2$ and $L=B^2$. Then:
$$V_2(\lambda K+\mu L)=\lambda^2+4\lambda\mu+\pi\mu^2$$
\end{example}
\only<1>{
\begin{center}
\begin{tikzpicture}
\coordinate (O) at (0,0);
\filldraw[fill=blue!50!red!10,draw=blue!50!red!50] (O) -- ++(2,0) arc (-90:0:1) -- ++(0,2) arc (0:90:1) -- ++(-2,0) arc (90:180:1) -- ++(0,-2) arc (180:270:1) --cycle;
\draw[blue!50!red!50] (O) -- ++(0,4);
\draw[blue!50!red!50] (O)++(2,0) -- ++(0,4);
\draw[blue!50!red!50] (O)++(-1,1) -- ++(4,0);
\draw[blue!50!red!50] (O)++(-1,1)++(0,2) -- ++(4,0);
\end{tikzpicture}
\end{center}
}
\only<2>{
\begin{theorem}
Let $K_1,\ldots,K_m\in\mathcal{K}_n$. Then, the Volume
$$V_n(\lambda_1K_1+\cdots+\lambda_mK_m)$$
is a homogeneous polynomial in $\lambda_1,\ldots,\lambda_m$
\end{theorem}
}
\end{frame}

\begin{frame}{Mixed Volumes}
\begin{definition}
Let $K_1,\ldots,K_n\in\mathcal{K}_n$, then their \alert{mixed Volume} $V(K_1,\ldots,K_n)$ is defined to be the coefficient of $\lambda_1\lambda_2\cdots\lambda_n$ of the polynomial $V_n(\lambda_1K_1+\cdots+\lambda_nK_n)$ over $n!$.
\end{definition}
\begin{corollary}
For any $K_1,\ldots,K_m\in\mathcal{K}_n$, we can write:
$$V_n(\lambda_1K_1+\cdots+\lambda_mK_m)=\sum_{i_1,\ldots,i_n\in[m]}V(K_{i_1},\ldots,K_{i_n})\lambda_{i_1}\cdots\lambda_{i_n}$$
\end{corollary}
\end{frame}

\begin{frame}{Some Properties}
\begin{description}[Multilinear]
\item[Symmetric] $V(K_1,K_2,\ldots,K_n)=V(K_2,K_1,\ldots,K_n)$
\item[Multilinear] $V(\lambda K_1+\mu K_1',K_2,\ldots,K_n)=\lambda V(K_1,K_2,\ldots,K_n)+\mu V(K_1',K_2,\ldots,K_n)$
\item[Nomalized] $V(K,K,\ldots,K)=V_n(K)$
\item[Monotone] $V(K_1,\ldots,K_n)\leq V(K_1',\ldots,K_n)$, if $K_1\subseteq K_1'$
\end{description}
\begin{example} In the example $K=[0,1]^2$, $L=B^2$ we had:
$$V(\lambda K+\mu L)=\lambda^2+4\lambda\mu+\pi\mu^2$$
And thus:
\pause
$$V(K,K)=1\qquad V(K,L)=V(L,K)=2\qquad V(L,L)=\pi$$
\end{example}
\end{frame}

\begin{frame}{A Special Case That Motivated Us}
We started at trying to compute $V_n(K+\varepsilon B^n)$.
\begin{definition} Let $K\in\mathcal{K}_n$ and $i\in\{0,\ldots,n\}$. The $i$-th \alert{Quermassintegral} of $K$ is defined to be:
$$W_i(K)=V(K,\ldots,K,B^n,\ldots,B^n)$$
where we have $n-i$ copies of $K$ and $i$ copies of $B^n$
\end{definition}
\begin{corollary}
$$V_n(K+rB^n)=\sum_{k=0}^n\binom{n}{k}W_i(K)r^i$$
\end{corollary}
\end{frame}

\begin{frame}{But How Can We Compute The Mixed Volume}
Using the inclusion-exclusion formula:
\begin{theorem} Let $K_1,\ldots,K_n\in\mathcal{K}_n$. Then:
$$V(K_1,\ldots,K_n)=\frac{1}{n!}\sum_{k=1}^n(-1)^{n+k}\sum_{i_1<\cdots<i_k}V_n(K_{i_1}+\cdots+K_{i_k})$$
\end{theorem}
\end{frame}

\begin{frame}{Mixed Subdivision}
We now make a detour through Polytopes
\begin{definition} Let $P_1,\ldots,P_m\in\mathcal{P}_n$. A \alert{subdivision} of $P_1+P_2+\cdots+P_m$ is a collection of cells $\mathcal{C}=\{F_1+F_2+\cdots+F_m:F_i\subseteq P_i\}$ partitioning the Minkowski sum $P_1+\cdots+P_n$, where each $F_i$ is a face of $P_i$.\\
A subdivision is called \alert{fine}, if it is generic enough.
\end{definition}
\pause
\begin{definition} Given $P_1,\ldots,P_n$ and $\mathcal{C}$ as above, a cell $C\in\mathcal{C}$ is called \alert{mixed}, if every face involved is of dimension $\leq1$.
\end{definition}
\end{frame}

\begin{frame}{The Mixed Volume Equals the Sum of Volumes of the Mixed Cells}
\begin{theorem}
Let $P_1,\ldots,P_n\in\mathcal{P}_n$, $\mathcal{C}$ a fine subdivision of $P_1+\cdots+P_n$ and $\mathcal{M}\subseteq\mathcal{C}$ the set of all mixed cells of $\mathcal{C}$. Then:
$$V(P_1,\ldots,P_n)=\sum_{C\in\mathcal{M}}V_n(C)$$
\end{theorem}
\begin{example}
\begin{center}
\begin{columns}
\column{.4\textwidth}
Let $P_1,P_2$ as in the picture, where the underlying lattice is $\mathbb{Z}^2$.
\column{.5\textwidth}
\begin{center}
\begin{tikzpicture}[scale=.5]
\coordinate (A) at (0,0);
\coordinate (B) at (-3,0);
\foreach\x in {0,1,2}
	\draw[gray,thin] (A)++(\x,-.2) -- ++(0,2.4);
\foreach\y in {0,1,2}
	\draw[gray,thin] (A)++(-.2,\y) -- ++(2.4,0);
\filldraw[fill=red!10, draw=red!50, thick] (A) -- ++(1,0) -- ++(0,1) -- ++(-1,0) -- cycle;
\node at (.5,.5) {$P_2$};
\node at (-.5,1) {$+$};
\foreach\x in {0,1,2}
	\draw[gray,thin] (B)++(\x,-.2) -- ++(0,2.4);
\foreach\y in {0,1,2}
	\draw[gray,thin] (B)++(-.2,\y) -- ++(2.4,0);
\filldraw[fill=blue!10, draw=blue!50, thick] (B) -- ++(1,0) -- ++(1,1) -- ++(-1,0) -- cycle;
\node at (-2,.5) {$P_1$};
\coordinate (AB) at (3,0);
\foreach\x in {0,1,2,3}
	\draw[gray,thin] (AB)++(\x,-.2) -- ++(0,2.4);
\foreach\y in {0,1,2}
	\draw[gray,thin] (AB)++(-.2,\y) -- ++(3.4,0);
\node at (2.5,1) {$=$};
\fill[fill=blue!50!red!10] (AB) -- ++(2,0) -- ++(1,1) -- ++(0,1) -- ++(-2,0) -- ++(-1,-1) -- cycle;
\draw[blue!50,thick] (AB) -- ++(1,0) -- ++(1,1) -- ++(-1,0) -- cycle;
\draw[red!50,thick] (AB)++(1,0)++(1,1) -- ++(1,0) -- ++(0,1) -- ++(-1,0) -- cycle;
\draw[blue!50,thick] (AB)++(0,1) -- ++(1,1) -- ++(1,0);
\draw[blue!50,thick] (AB)++(2,0) -- ++(1,1);
\draw[red!50,thick] (AB) -- ++(0,1);
\draw[red!50,thick] (AB)++(1,1) -- ++(0,1);
\draw[red!50,thick] (AB)++(1,0) -- ++(1,0);
\foreach\p in {(0,0),(1,0),(2,0),(0,1),(1,1),(2,1),(3,1),(1,2),(2,2),(3,2)}
	\fill[fill=blue!50!red!50] (AB)++\p circle (.7pt);
\end{tikzpicture}
\end{center}
\end{columns}
\end{center}
Then $V(P_1,P_2)=3$, because there are $3$ mixed cells in this decomposition, all having volume $1$.
\end{example}
\end{frame}

\begin{frame}{Finding Some Fine Subdivision}
\begin{theorem}
Let $P_1,\ldots,P_n\in\mathcal{P}_n$ generic enough polytopes and $\phi_1,\ldots,\phi_n:\mathbb{R}^n\to\mathbb{R}$ generic enough affine maps. The surface of $\phi_1P_1+\cdots+\phi_nP_n$ is naturally partitioned in some cells $\tilde{\mathcal{C}}$. Then, the lower (resp. upper) cell subdivision
$$\mathcal{C}:=\left\{p(C):C\in\tilde{\mathcal{C}}^-\right\}$$
is a fine subdivision of $P_1+\cdots+P_n$, where $\tilde{\mathcal{C}}^-$ are the cells subdividing the ``lower'' part of the surface.
\end{theorem}
\end{frame}

%\documentclass{beamer}
%\mode<presentation>
%\usetheme{Frankfurt}
%
%\usepackage{tikz}
%\usepackage{tikz-3dplot}
%\usetikzlibrary{positioning,calc}
%
%\begin{document}
\begin{frame}[fragile]{Projecting (Lowering) the Mixed Volume}
\begin{center}
\only<1-8>\tdplotsetmaincoords{0}{0}
% 3d rotation
\only<2-7>{\tdplotsetmaincoords{70}{15}}
\begin{tikzpicture}[tdplot_main_coords,
vrtx/.style={fill=#1!50},
edge/.style={draw=#1!50, thick},
face/.style={fill=#1!30, fill opacity=.7},
both/.style={edge=#1, face=#1},
grid/.style={gray!50,thin},
lift/.style={draw=brown}]
% colors
\definecolor{cP}{rgb}{1,.5,.5};
\definecolor{cQ}{rgb}{.5,.5,1};
\colorlet{cPQ}{cP!50!cQ};
% nmb of vcs
\def\nP{4};
\def\nQ{4};
% definition of P,Q (relative)
\begin{scope}[tdplot_screen_coords]
\foreach\x/\y[count=\i] in {0/0,1/0,1/1,0/1}
	\coordinate (P\i) at (\x,\y);
\foreach\x/\y[count=\i] in {1/0,2/1,1.5/1.5,0/1}
	\coordinate (Q\i) at (\x,\y);
\end{scope}
% basepoints
\coordinate (bP) at (-2,0,0);
\coordinate (bQ) at (0,0,0);
\coordinate (bPQ) at (3,0,0);
% affine maps
\def\hP{3}
\def\lP(#1,#2){-.8*#1+.3*#2};
\def\hQ{3.5}
\def\lQ(#1,#2){-.7*#2};%{.5*#1+.5*#2};
\def\hPQ{3.5}
% grid
\def\maxP{1};
\def\mayP{1};
\def\maxQ{2};
\def\mayQ{2};
\pgfmathsetmacro\maxPQ{\maxP+\maxQ};
\pgfmathsetmacro\mayPQ{\mayP+\mayQ};
\foreach\name in {P,Q,PQ}{
	\expandafter\let\expandafter\max\csname max\name\endcsname;
	\expandafter\let\expandafter\may\csname may\name\endcsname;
	\foreach\x in {0,...,\max}
		\draw[grid] ($(b\name)+(\x,-.3,0)$) -- ($(b\name)+(\x,\may+.3,0)$);
	\foreach\y in {0,...,\may}
		\draw[grid] ($(b\name)+(-.3,\y,0)$) -- ($(b\name)+(\max+.3,\y,0)$);
}

%%%%% From this point it should work autmatically
%%%%% for small perturbations
% define (bP*),(bQ*)
\foreach\name in {P,Q}{
	\expandafter\let\expandafter\n\csname n\name\endcsname;
	\foreach\i in {1,...,\n}
		\path let \p1=(\name\i),\n1={\x1*1pt/1cm},\n2={\y1*1pt/1cm} in
		coordinate (b\name\i) at
		($(b\name)+(\n1,\n2,0)$);
}
% define (zbP*),(zbQ*) [lifts of P,Q]
\foreach\name in {P,Q}{
	\expandafter\let\expandafter\n\csname n\name\endcsname;
	\expandafter\let\expandafter\l\csname l\name\endcsname;
	\expandafter\let\expandafter\h\csname h\name\endcsname;
	\foreach\i in {1,...,\n}{
		\path let \p1=(\name\i),\n1={\x1*1pt/1cm},\n2={\y1*1pt/1cm} in
		coordinate (zb\name\i)
		at ($(b\name\i)+(0,0,\l(\n1,\n2)+\h)$);
	}
}
% define (bPQ**)
\foreach\i in {1,...,\nP}{
	\foreach\j in {1,...,\nQ}
		\path let \p1=(P\i),\p2=(Q\j),\n1={\x1*1pt/1cm},\n2={\y1*1pt/1cm},\n3={\x2*1pt/1cm},\n4={\y2*1pt/1cm} in
		coordinate (bPQ\i\j) at
		($(bPQ)+(\n1+\n3,\n2+\n4,0)$);
}
% define (zbPQ**) [lift of PQ]
\foreach\i in {1,...,\nP}{
	\foreach\j in {1,...,\nQ}{
		\path let \p1=(P\i),\p2=(Q\j),\n1={\x1*1pt/1cm},\n2={\y1*1pt/1cm},\n3={\x2*1pt/1cm},\n4={\y2*1pt/1cm} in
		coordinate (zbPQ\i\j) at
		($(bPQ\i\j)+(0,0,{\lP(\n1,\n2)+\lQ(\n3,\n4)+\hPQ})$);
	}
}

%%%%% START of draw commands
% draw (bP*),(bQ*)
\filldraw<1-8>[both=cP] (bP1) -- (bP2) -- (bP3) -- (bP4) -- cycle;
\filldraw<1-8>[both=cQ] (bQ1) -- (bQ2) -- (bQ3) -- (bQ4) -- cycle;
% draw lifted +/=
\node<1> at ($.25*(bP)+.75*(bQ)+(0,.5,0)$) {$+$};
\node<1> at ($.15*(bQ)+.85*(bPQ)+(0,.5,0)$) {$=$};
% draw (bPQ**)
\draw<1-5>[both=cPQ] (bPQ11)--(bPQ21)--(bPQ22)--(bPQ32)--(bPQ33)--(bPQ43)--(bPQ44)--(bPQ14)--cycle;

\only<3-6>{
	% draw lifts for P
	\foreach\i in {1,...,\nP}{
		\draw[lift] (bP\i)--(zbP\i);
	}
	% draw (zbP*)
	\filldraw[both=cP] (zbP1) -- (zbP2) -- (zbP3) -- (zbP4) -- cycle;
	% draw (zbQ*)
	\filldraw[both=cQ] (zbQ1) -- (zbQ2) -- (zbQ3) -- (zbQ4) -- cycle;
	% draw lifts for Q
	\foreach\i in {1,...,\nQ}{
		\draw[lift] (bQ\i)--(zbQ\i);
	}
}
\only<4-6>{
	% draw lifted +/=
	\node<4> at ($.1*(bP)+.9*(bQ)+.5*(0,.5,\hP+\hQ)$) {$+$};
	\node<4> at ($.1*(bQ)+.9*(bPQ)+.5*(0,.5,\hP+\hQ)$) {$=$};
	% draw the back lift lines
	\foreach\i/\j in {1/4,2/4,3/4,4/4,4/3,3/3,3/2,2/3,2/2}{
		\draw<6>[lift] (bPQ\i\j)--(zbPQ\i\j);
	}
	% draw lifted faces under
	\filldraw[both=cP] (zbPQ14)--(zbPQ24)--(zbPQ34)--(zbPQ44)--cycle;
	\foreach\i in {1,2,3,4}{
		\fill[vrtx=cPQ] (zbPQ\i4) circle (1pt);
	}
	\foreach\ia/\ib/\ja/\jb in {3/4/3/4,2/3/3/4,2/3/2/3,1/2/4/1}{
		\fill[face=cPQ] (zbPQ\ia\ja)--(zbPQ\ib\ja)--(zbPQ\ib\jb)--(zbPQ\ia\jb)--cycle;
		\draw[edge=cQ] (zbPQ\ia\ja)--(zbPQ\ia\jb);
		\draw[edge=cQ] (zbPQ\ib\ja)--(zbPQ\ib\jb);
		\draw[edge=cP] (zbPQ\ia\ja)--(zbPQ\ib\ja);
		\draw[edge=cP] (zbPQ\ia\jb)--(zbPQ\ib\jb);
		\foreach\i in {\ia,\ib}{
			\foreach\j in {\ja,\jb}{
				\fill[vrtx=cPQ] (zbPQ\i\j) circle (1pt);
			}
		}
	}
	\filldraw[both=cQ] (zbPQ21)--(zbPQ22)--(zbPQ23)--(zbPQ24)--cycle;
	\foreach\j in {1,2,3,4}{
		\fill[vrtx=cPQ] (zbPQ2\j) circle (1pt);
	}
	% draw lift of the boundary ob (bPQ**)
	\draw<5-6> (zbPQ11)--(zbPQ21)--(zbPQ22)--(zbPQ32)--(zbPQ33)--(zbPQ43)--(zbPQ44)--(zbPQ14)--cycle;
	% draw lifted faces above
	\filldraw<4>[both=cQ] (zbPQ41)--(zbPQ42)--(zbPQ43)--(zbPQ44)--cycle;
	\foreach\j in {1,2,3,4}{
		\fill<4>[vrtx=cPQ] (zbPQ4\j) circle (1pt);
	}
	\foreach\ia/\ib/\ja/\jb in {4/1/4/1,3/4/2/3,4/1/1/2,1/2/1/2}{
		\fill<4>[face=cPQ] (zbPQ\ia\ja)--(zbPQ\ib\ja)--(zbPQ\ib\jb)--(zbPQ\ia\jb)--cycle;
		\draw<4>[edge=cQ] (zbPQ\ia\ja)--(zbPQ\ia\jb);
		\draw<4>[edge=cQ] (zbPQ\ib\ja)--(zbPQ\ib\jb);
		\draw<4>[edge=cP] (zbPQ\ia\ja)--(zbPQ\ib\ja);
		\draw<4>[edge=cP] (zbPQ\ia\jb)--(zbPQ\ib\jb);
		\foreach\i in {\ia,\ib}{
			\foreach\j in {\ja,\jb}{
				\fill<4>[vrtx=cPQ] (zbPQ\i\j) circle (1pt);
			}
		}
	}
	\filldraw<4>[both=cP] (zbPQ12)--(zbPQ22)--(zbPQ32)--(zbPQ42)--cycle;
	\foreach\i in {1,2,3,4}{
		\fill<4>[vrtx=cPQ] (zbPQ\i2) circle (1pt);
	}
	% draw missing lift lines
	\foreach\i/\j in {1/1,2/1}{
		\draw<6>[lift] (bPQ\i\j)--(zbPQ\i\j);
	}
}
% draw projections
\filldraw<6->[both=cP] (bPQ14)--(bPQ24)--(bPQ34)--(bPQ44)--cycle;
\foreach\ia/\ib/\ja/\jb in {3/4/3/4,2/3/3/4,2/3/2/3,1/2/4/1}{
	\fill<6->[face=cPQ] (bPQ\ia\ja)--(bPQ\ib\ja)--(bPQ\ib\jb)--(bPQ\ia\jb)--cycle;
	\draw<6->[edge=cQ] (bPQ\ia\ja)--(bPQ\ia\jb);
	\draw<6->[edge=cQ] (bPQ\ib\ja)--(bPQ\ib\jb);
	\draw<6->[edge=cP] (bPQ\ia\ja)--(bPQ\ib\ja);
	\draw<6->[edge=cP] (bPQ\ia\jb)--(bPQ\ib\jb);
}
\filldraw<6->[both=cQ] (bPQ21)--(bPQ22)--(bPQ23)--(bPQ24)--cycle;
\foreach\i/\j in {1/4,2/4,3/4,4/4,2/1,2/2,2/3,1/1,3/2,3/3,4/3}{
	\fill<6->[vrtx=cPQ] (bPQ\i\j) circle (1pt);
}

%%%%% ONLY for debug purposes
%%draw colored edges inside (zbPQ**)
%\foreach\i in {1,...,\nP}{
%	\draw[edge=cQ] (zbPQ\i1) -- (zbPQ\i2) -- (zbPQ\i3) -- (zbPQ\i4) -- cycle;
%}
%\foreach\j in {1,...,\nQ}{
%	\draw[edge=cP] (zbPQ1\j) -- (zbPQ2\j) -- (zbPQ3\j) -- (zbPQ4\j) -- cycle;
%}
%%draw vertices inside (zbPQ**)
%\foreach\i in {1,...,\nP}{
%	\foreach\j in {1,...,\nQ}{
%		\draw[lift] (bPQ\i\j)--(zbPQ\i\j);
%		\fill[vrtx=cPQ] (zbPQ\i\j) circle (2pt);
%	}
%}
\end{tikzpicture}
\end{center}
\end{frame}
%\end{document}


\end{document}
